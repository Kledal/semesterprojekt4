\documentclass[a4paper,11pt,fleqn,dvipsnames,twoside,openany]{memoir} 	% Openright aabner kapitler paa hoejresider (openany begge)


%%%% Basis sideopsætning %%%%
\usepackage{longtable}
\usepackage{lmodern} 						% Skrifttype
\usepackage{fullpage}						% Bruger hele siden
\usepackage[utf8]{inputenc}					% Input-indkodning af tegnsaet (UTF8)
\usepackage[english]{babel}					% Dokumentets sprog
\usepackage[T1]{fontenc}						% Output-indkodning af tegnsaet (T1)
\usepackage{ragged2e,anyfontsize}			% Justering af elementer
\usepackage{fixltx2e}						% Retter forskellige fejl i LaTeX-kernen
\chapterstyle{section}
\usepackage{url}
\usepackage{siunitx}
\usepackage{csquotes}
\usepackage[
backend=biber,
style=authoryear,
citestyle=authoryear-ibid,
]{biblatex}

\usepackage{chngcntr}
%%%% Header %%%%
\addbibresource{Formalia/References.bib}





%%%% Tabeller med venstre tekst %%%%
\usepackage{array}
\newcolumntype{L}[1]{>{\raggedright\let\newline\\\arraybackslash\hspace{0pt}}m{#1}}
\newcolumntype{C}[1]{>{\centering\let\newline\\\arraybackslash\hspace{0pt}}m{#1}}
\newcolumntype{R}[1]{>{\raggedleft\let\newline\\\arraybackslash\hspace{0pt}}m{#1}}


%%%% Figurer og tabeller (floats) %%%%
\usepackage{graphicx} 						% Haandtering af eksterne billeder (JPG, PNG, EPS, PDF)
%\usepackage{eso-pic}						% Tilfoej billedekommandoer paa hver side
%\usepackage{wrapfig}						% Indsaettelse af figurer omsvoebt af tekst. \begin{wrapfigure}{Placering}{Stoerrelse}
\usepackage{multirow}                		% Fletning af raekker og kolonner (\multicolumn og \multirow)
\usepackage{multicol}         	     	   	% Muliggoer output i spalter
\usepackage{rotating}						% Rotation af tekst med \begin{sideways}...\end{sideways}
\usepackage{colortbl} 						% Farver i tabeller (fx \columncolor og \rowcolor)
\usepackage{xcolor}							% Definer farver med \definecolor. Se mere: http://en.wikibooks.org/wiki/LaTeX/Colors
\usepackage{flafter}							% Soerger for at floats ikke optraeder i teksten foer deres reference
\let\newfloat\relax 							% Justering mellem float-pakken og memoir
\usepackage{float}							% Muliggoer eksakt placering af floats, f.eks. 

% ¤¤ Matematik mm. ¤¤
\usepackage{amsmath,amssymb,stmaryrd} 		% Avancerede matematik-udvidelser
\usepackage{mathtools}						% Andre matematik- og tegnudvidelser
\usepackage{textcomp}                 		% Symbol-udvidelser (f.eks. promille-tegn med \textperthousand )
\usepackage{rsphrase}						% Kemi-pakke til RS-saetninger, f.eks. \rsphrase{R1}
\usepackage[version=3]{mhchem} 				% Kemi-pakke til flot og let notation af formler, f.eks. \ce{Fe2O3}
\usepackage{siunitx}						% Flot og konsistent praesentation af tal og enheder med \si{enhed} og \SI{tal}{enhed}
\sisetup{output-decimal-marker = {,}}		% Opsaetning af \SI (DE for komma som decimalseparator) 

% ¤¤ Referencer og kilder ¤¤ %
\usepackage[danish]{varioref}				% Muliggoer bl.a. krydshenvisninger med sidetal (\vref)
%\usepackage{natbib}							% Udvidelse med naturvidenskabelige citationsmodeller
%\usepackage{xr}							% Referencer til eksternt dokument med \externaldocument{<NAVN>}
%\usepackage{glossaries}					% Terminologi- eller symbolliste (se mere i Daleifs Latex-bog)

% ¤¤ Misc. ¤¤ %
\usepackage{listings}						% Placer kildekode i dokumentet med \begin{lstlisting}...\end{lstlisting}
\usepackage{lipsum}							% Dummy text \lipsum[..]
\usepackage[shortlabels]{enumitem}			% Muliggoer enkelt konfiguration af lister
\usepackage{pdfpages}						% Goer det muligt at inkludere pdf-dokumenter med kommandoen \includepdf[pages={x-y}]{fil.pdf}	
\pdfoptionpdfminorversion=6					% Muliggoer inkludering af pdf dokumenter, af version 1.6 og hoejere
\pretolerance=2500 							% Justering af afstand mellem ord (hoejt tal, mindre orddeling og mere luft mellem ord)

% Kommentarer og rettelser med \fxnote. Med 'final' i stedet for 'draft' udloeser hver note en error i den faerdige rapport.
\usepackage[footnote,draft,danish,silent,nomargin]{fixme}	

% ¤¤ Litteraturlisten ¤¤ %
%\bibpunct[,]{[}{]}{;}{a}{,}{,} 				% Definerer de 6 parametre ved Harvard henvisning (bl.a. parantestype og seperatortegn)
%\bibliographystyle{bibtex/harvard}			% Udseende af litteraturlisten.
\bibliography{../References}
% ¤¤ Indholdsfortegnelse ¤¤ %
\setsecnumdepth{subsubsection}		 			% Dybden af nummerede overkrifter (part/chapter/section/subsection)
\maxsecnumdepth{subsubsection}					% Dokumentklassens graense for nummereringsdybde
\settocdepth{subsection} 					% Dybden af indholdsfortegnelsen

% ¤¤ Visuelle referencer ¤¤ %
\usepackage[colorlinks]{hyperref}			% Danner klikbare referencer (hyperlinks) i dokumentet.
\hypersetup{colorlinks = true,				% Opsaetning af farvede hyperlinks (interne links, citeringer og URL)
    linkcolor = black,
    citecolor = black,
    urlcolor = black
}

% ¤¤ Opsaetning af figur- og tabeltekst ¤¤ %
\captionnamefont{\small\bfseries\itshape}	% Opsaetning af tekstdelen ('Figur' eller 'Tabel')
\captiontitlefont{\small}					% Opsaetning af nummerering
\captiondelim{. }							% Seperator mellem nummerering og figurtekst
\hangcaption							 	% Venstrejusterer flere-liniers figurtekst under hinanden
\captionwidth{\linewidth}					% Bredden af figurteksten
\setlength{\belowcaptionskip}{0pt}			% Afstand under figurteksten
		
% ¤¤ Opsaetning af listings ¤¤ %

\definecolor{commentGreen}{RGB}{34,139,24}
\definecolor{stringPurple}{RGB}{208,76,239}

\lstset{language=c++,					% Sprog
	basicstyle=\ttfamily\scriptsize,		% Opsaetning af teksten
	keywords={for,if,while,else,elseif,		% Noegleord at fremhaeve
			  end,break,return,case,
			  switch,function},
	keywordstyle=\color{blue},				% Opsaetning af noegleord
	commentstyle=\color{commentGreen},		% Opsaetning af kommentarer
	stringstyle=\color{stringPurple},		% Opsaetning af strenge
	showstringspaces=false,					% Mellemrum i strenge enten vist eller blanke
	numbers=left, numberstyle=\tiny,		% Linjenumre
	extendedchars=true, 					% Tillader specielle karakterer
	columns=flexible,						% Kolonnejustering
	breaklines, breakatwhitespace=true,		% Bryd lange linjer
}


